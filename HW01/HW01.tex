\documentclass[12pt,a4paper]{article}
\usepackage[UTF8]{ctex}     %先引入ctex
\usepackage[utf8]{inputenc} %再引入inputenc
\usepackage{graphicx}
% \usepackage{lazylatex}
% \tcbuselibrary{documentation}
\usepackage{amsmath}
\usepackage{bookmark}
\usepackage{enumerate}
\usepackage{geometry}
\graphicspath{{img/}}
% 边距
\geometry{left=2.0cm,right=2.0cm,top=2.0cm,bottom=3.0cm}
% 大题
\newenvironment{problems}{\begin{list}{}{\renewcommand{\makelabel}[1]{\textbf{##1}\hfil}}}{\end{list}}
% 小题
\newenvironment{steps}{\begin{list}{}{\renewcommand{\makelabel}[1]{##1.\hfil}}}{\end{list}}
% 答
\providecommand{\ans}{\textbf{答}:~}
% 解
\providecommand{\sol}{\textbf{解}.~}

\begin{document}
\title{\normalsize \underline{编译原理(A)}\\\LARGE第 1 次作业}
\author{Log Creative }
\date{\today}
\maketitle

\begin{problems}
    \item[1] 如果一个程序设计语言是通用的,那么它的编译程序或解释程序是否是通用的,为什么?
     
    \ans \textbf{不正确。}对于不同的计算机体系结构,通常需要针对不同的指令集(CISC, RISC等)采用不同的汇编代码生成器,以及在不同的操作系统(Windows, macOS, linux等)上需要使用能够在该操作系统上能够运行的编译器。而一个程序设计语言是通用的主要指文法通用,在不同的机器上有对应的编译器生成对应的目标代码。
    \item[2] 借鉴编译器的移植构造思想说明 Java 语言是如何实现跨平台处理的。
    
    \ans 编译器的移植构造思想表明,我们只需要在目标机器上有编译器 $C^{M,A}_{A}$ ,就可以将 $C^{J,A}_{M}$ 通过上述编译器变为 $C^{J,A}_{A}$,即
    \begin{equation*}
        C^{J,A}_{M}\rightarrow \fbox{$C^{M,A}_{A}$}\rightarrow C^{J,A}_{A}
    \end{equation*}

    Java 程序($J$)以 \textbf{Java 字节码}($M$)的方式分发,然后在对应的机器上,要么被解释执行,要么被动态地(运行时)编译为本地代码($A$),以跨平台处理。
\end{problems}


\end{document}
