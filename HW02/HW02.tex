\documentclass[12pt,a4paper]{article}
\usepackage[UTF8]{ctex}     %先引入ctex
\usepackage[utf8]{inputenc} %再引入inputenc
\usepackage{graphicx}
% \usepackage{lazylatex}
% \tcbuselibrary{documentation}
\usepackage{amsmath}
\usepackage{bookmark}
\usepackage{enumerate}
\usepackage{geometry}
\usepackage{tikz}
\usepackage{forest}

\graphicspath{{img/}}
% 边距
\geometry{left=2.0cm,right=2.0cm,top=2.0cm,bottom=3.0cm}
% 大题
\newenvironment{problems}{\begin{list}{}{\renewcommand{\makelabel}[1]{\textbf{##1}\hfil}}}{\end{list}}
% 小题
\newenvironment{steps}{\begin{list}{}{\renewcommand{\makelabel}[1]{##1.\hfil}}}{\end{list}}
% 答
\providecommand{\ans}{\textbf{答}:~}
% 解
\providecommand{\sol}{\textbf{解}.~}

\begin{document}
\title{\normalsize \underline{编译原理(A)}\\\LARGE第 2 次作业}
\author{Log Creative }
\date{\today}
\maketitle

\begin{problems}
    \item[1] 写出不以 0 开头的奇数的上下文无关文法。
    
    \ans 
    \begin{align*}
        odd &\rightarrow  Z|XZ|XYZ \\
        Z &\rightarrow \texttt{1|3|5|7|9}\\
        X &\rightarrow Z|\texttt{2|4|6|8}\\
        Y &\rightarrow YY\texttt{|0|}X\\
    \end{align*}
    \item[2] 设 $param$ 为 C++ 语言的实际参数,小写字母 a$\cdots$z 和数字 0$\cdots$9 可用的符号,参考例 2.3 和例 2.4 写出 C++ 函数调用的完整的上下文无关文法。
    
    \ans 
    \begin{align*}
        call &\rightarrow \textbf{id}(optparams)\\
        optparams &\rightarrow params | \epsilon \\
        params &\rightarrow params, param | param\\
        \textbf{id} &\rightarrow L|LS\\
        S &\rightarrow T|ST\\
        T &\rightarrow L|D\\
        L &\rightarrow \texttt{a|b|}\cdots\texttt{|z}\\
        D &\rightarrow \texttt{0|1|2|3|4|5|6|7|8|9}
    \end{align*}
    \item[3] 构建一个语法制导翻译方案,该方案把算术表达式从中缀表示方法翻译成前缀表示方法,并给出 9-5+2 的注释分析树。
    
    \ans 
    从中缀表示方法翻译成前缀表示的语法制导翻译方案:
    \begin{align*}
        expr \rightarrow& \text{ \{print(\texttt{'+'})\}}  &&expr_1 + term \\
        expr \rightarrow& \text{ \{print(\texttt{'-'})\}}  &&expr_1 - term \\
        expr \rightarrow& && term \\      
        term \rightarrow& \text{ \{print(\texttt{'0'})\}}  && \texttt{0}\\
        term \rightarrow& \text{ \{print(\texttt{'1'})\}}  && \texttt{1}\\
                        & \cdots  && \\
        term \rightarrow& \text{ \{print(\texttt{'9'})\}}  && \texttt{9}\\
    \end{align*}

    注释分析树:
    \begin{figure}[h]
        \centering
        \begin{forest}
        [\texttt{expr.t=+-952}
            [\texttt{expr.t=-95}
                [\texttt{expr.t=9}[\texttt{term.t=9}[\texttt{9}]]]
                [\texttt{-}]
                [\texttt{term.t=5}[\texttt{5}]]
            ]
            [\texttt{+}]
            [\texttt{term.t=2}[\texttt{2}]]
        ]
    \end{forest}
    \end{figure}
    
\end{problems}


\end{document}
