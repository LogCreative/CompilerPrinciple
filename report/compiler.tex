\documentclass[UTF8]{ctexart}
\usepackage{amsmath,amssymb}
\newtheorem{definition}{定义}
\usepackage[colorlinks]{hyperref}
\usepackage{booktabs}
\usepackage{array}
\usepackage{float}
\bibliographystyle{IEEEtran}
\title{\normalsize\underline{编译原理(A)大作业}\\\Large 算符优先分析表和卷积优化}
\author{李子龙 518070910095}

\begin{document}
    \maketitle
    \tableofcontents
    \clearpage

\part{算符优先分析表}
\subsection*{程序输出}
% \input{output1.txt}
% \input{output2.txt}

\section{题目分析}

首先需要声明在算符优先语法中算符优先级的定义\cite{Floyd1963}。
\begin{definition}\label{def:op}
    对于两个终结符 $T_1$ 和 $T_2$,有下面的算符优先级定义(其中 $U_1$ 是非终结符)
    \begin{enumerate}
        \item $T_1=T_2$ 如果存在产生式 $U\rightarrow xT_1T_2y$ 或 $U\rightarrow xT_1 U_1 T_2 y$。
        \item $T_1<T_2$ 如果存在产生式 $U\rightarrow xT_1U_1y$ 而且存在一个推导 $U_1\Rightarrow z$ 使得 $T_2$ 是 $z$ 的最左终结符。
        \item $T_1>T_2$ 如果存在产生式 $U\rightarrow xU_1T_2y$ 而且存在一个推导 $U_1\Rightarrow z$ 使得 $T_1$ 是 $z$ 的最右终结符。
    \end{enumerate}
\end{definition}

本部分即针对一个上下文无关文法,输出算符优先分析表。如果文法是有二义性的,将会报错。如果两个终结符之间没有上述三个关系的其中一个,将会留空,意为没有优先关系。

根据定义 \ref{def:op},可以对符号构造下面两个集合以判断情况 2 和 3:

\begin{definition}\label{def:vt}
    假设 $V_T$ 是该文法终结符号对应的集合,$V_N$ 是该文法非终结符号对应的集合。对符号 $U_1$ 定义下面两个集合:
    \begin{align}
        \textit{FIRSTVT}(U_1) &= \{T|(U_1\Rightarrow Ty \vee U_1\Rightarrow U_2Ty)\land T\in V_T\land U_2\in V_N\}  \\
        \textit{LASTVT}(U_1) &= \{T|(U_1\Rightarrow xT \vee U_1\Rightarrow xTU_2)\land T\in V_T\land U_2\in V_N\}
    \end{align}
\end{definition}

除了上面的定义方法,对于这样的集合,还有这样的性质:
\begin{align}
    (U_1\Rightarrow U_2y) &\Rightarrow \textit{FIRSTVT}(U_2)\subseteq \textit{FIRSTVT}(U_1) \\
    (U_1\Rightarrow xU_2) &\Rightarrow \textit{LASTVT}(U_2)\subseteq \textit{LASTVT}(U_1)
\end{align}

这样定义 \ref{def:op} 就有了如下的等价定义:
\begin{definition}\label{def:opn}
    对于两个终结符 $T_1$ 和 $T_2$,有下面的算符优先级定义(其中 $U_1$ 是非终结符)
    \begin{enumerate}
        \item 如果找到了这样的产生式右部 $T_1T_2$ 或 $T_1U_1T_2$,那么 $T_1=T_2$。
        \item 如果找到了这样的产生式右部 $T_1U_1$ 且 $T_2\in\textit{FIRSTVT}(U_1)$,那么 $T_1<T_2$。
        \item 如果找到了这样的产生式右部 $U_1T_2$ 且 $T_1\in\textit{LASTVT}(U_1)$,那么 $T_1>T_2$。
    \end{enumerate}
\end{definition}

\section{算法实现}



%% 二义性
% 如果一个文法是无二义性的,那么该文法的每个右句型都有且只有一个句柄。

\part{卷积优化}
\subsection*{优化结果}
\begin{table}[H]
    \begin{tabular}{m{17em}rrc}
        \toprule
        输入大小和输出大小 & 优化前 & 优化后 & 提升效率 \\
        \midrule
        \ttfamily
        n, ic, ih, iw = 1, 3, 32, 32~~~~~
        oc, kh, kw = 32, 3, 3
        & & & \\
        \hline
        \ttfamily
        n, ic, ih, iw = 100, 512, 32, 32
        oc, kh, kw = 1024, 3, 3
        & & & \\
        \bottomrule
    \end{tabular}
\end{table}

\bibliography{ref}

\end{document}